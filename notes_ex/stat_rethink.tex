%%%%%%%%%%%%%%%%%%%%%%%%%%%%%%%%%%%%%%%%%%%%%%%%%%%%%%%%%%%%%%%%%%%%%%%%%%%%%%%
% In order to compile this file use:
% xelatex exercises.tex
%%%%%%%%%%%%%%%%%%%%%%%%%%%%%%%%%%%%%%%%%%%%%%%%%%%%%%%%%%%%%%%%%%%%%%%%%%%%%%%
\documentclass[12pt]{report}

%-------------------------------------------------------
% THEMES, COLORS
%-------------------------------------------------------
\usepackage[top=2.5cm, bottom=3cm, left=2.5cm, right=2.5cm]{geometry}
\usepackage{amsmath,mathtools,bm,amsthm}
\usepackage{underscore}
\usepackage{xcolor}
\usepackage{graphicx}
\usepackage{algpseudocode}
\usepackage{rsfso} % for \mathscr
%--------------------------------------------------------
% COLOR DEFINITIONS
%---------------------------------------------------------
\definecolor{amber}{rgb}{1.0, 0.49, 0.0}
\definecolor{olivedrab}{rgb}{0.42, 0.56, 0.14}
\definecolor{darkolivegreen}{rgb}{0.33, 0.42, 0.18}
\definecolor{chromeyellow}{rgb}{1.0, 0.65, 0.0}
\definecolor{beige}{rgb}{0.96, 0.96, 0.86}
\definecolor{bluegreen}{rgb}{3, 166, 155}
\definecolor{pitchblack}{rgb}{0, 0, 0}
\definecolor{lightbeige}{rgb}{255, 251, 241}
\definecolor{mediumgray}{rgb}{183, 183, 183}
\definecolor{carrotorange}{rgb}{0.93, 0.57, 0.13}
\definecolor{orange-red}{rgb}{1.0, 0.27, 0.0}
\definecolor{ferngreen}{rgb}{0.31, 0.47, 0.26}
\definecolor{darkspringgreen}{rgb}{0.09, 0.45, 0.27}
\definecolor{cocoabrown}{rgb}{0.82, 0.41, 0.12}
\definecolor{darkorange}{rgb}{1.0, 0.55, 0.0}
\definecolor{deepcarrotorange}{rgb}{0.91, 0.41, 0.17}
%---------------------------------------------------------

%-------------------------------------------------------------------------
% NEW THEME
%-------------------------------------------------------------------------
\definecolor{UBCblue}{rgb}{0.04706, 0.13725, 0.26667} % UBC Blue (primary)
\definecolor{UBCgrey}{rgb}{0.3686, 0.5255, 0.6235} % UBC Grey (secondary)

%\usepackage{helvet}
%-------------------------------------------------------
% DEFFINING AND REDEFINING COMMANDS
%-------------------------------------------------------
\newcommand\var{\texttt}
\usepackage{amssymb}
\renewcommand{\emptyset}{\varnothing}

%-----------------------------------------------------------------------------
% FONTS TRIED
%-----------------------------------------------------------------------------
%\usepackage{ccfonts}
%\usepackage{mathpazo}
%\usepackage{newtxtext}
%\usepackage{newtxmath}
%-----------------------------------------------------------------------------
% FONT PACKAGES
%-----------------------------------------------------------------------------

\usepackage[utf8]{inputenc}
\usepackage[english]{babel}
\usepackage[T1]{fontenc}
\usepackage{fontspec}

%-------------------------------------------------------
% INFORMATION IN THE TITLE PAGE
%-------------------------------------------------------


\newtheorem{lemma}{Lemma}[chapter]
\newtheorem{theorem}{Theorem}[chapter]
\newtheorem{corollary}{Corollary}[chapter]
\theoremstyle{definition}
\newtheorem{definition}{Definition}[chapter]
\newtheorem{example}{Example}[chapter]
\newtheorem{exer}{Exercise}[chapter]
\newtheorem*{note}{Note}
\newtheorem*{solution}{Solution}


\newcommand{\states}{\ensuremath{\mathcal{S}}}
\newcommand{\actions}{\ensuremath{\mathcal{A}}}
\newcommand{\rewards}{\ensuremath{\mathcal{R}}}
\newcommand{\discount}{\ensuremath{\gamma}}
\newcommand{\policy}{\ensuremath{\pi}}

\newcommand{\vcdim}{\ensuremath{\text{VCdim}}}
\newcommand{\algo}{\ensuremath{\mathcal{A}}}
\newcommand{\hypclass}{\ensuremath{\mathcal{H}}}
\newcommand{\parity}[1]{\ensuremath{\mathcal{H}_{#1\text{-parity}}}}
\newcommand{\rect}[1]{\ensuremath{\mathcal{H}_{\text{rect}}^{#1}}}
\newcommand{\conj}[1]{\ensuremath{\mathcal{H}_{\text{con}}^{#1}}}
\newcommand{\mconj}[1]{\ensuremath{\mathcal{H}_{\text{mcon}}^{#1}}}
\newcommand{\zeroone}{\ensuremath{\operatorname{0-1}}}
\newcommand{\dom}{\ensuremath{\mathcal{X}}}
\newcommand{\range}{\ensuremath{\mathcal{Y}}}
\newcommand{\Nat}{\ensuremath{\mathbf{N}}}
\newcommand{\dist}{\ensuremath{\mathcal{D}}}
\newcommand{\R}{\ensuremath{\mathbf{R}}}
\newcommand{\Rone}{\ensuremath{\mathbf{R}}}
\newcommand{\Rpos}{\ensuremath{\mathbf{R}_{+}}}
\newcommand{\Rtwo}{\ensuremath{\mathbf{R}^2}}
\newcommand{\Prtwo}[2]{\mathbf{Pr}_{#1} \left \{ #2 \right \}}
\newcommand{\Prone}[1]{\mathbf{Pr} \left \{ #1 \right \}}
\newcommand{\Exptwo}[2]{\mathbf{E}_{#1} \left [ #2 \right ]}
\newcommand{\Expone}[1]{\mathbf{E} \left [ #1 \right ]}
\newcommand{\ceiling}[1]{\ensuremath{\left \lceil #1 \right \rceil}}
\newcommand{\angular}[1]{\ensuremath{\left \langle #1 \right \rangle}}
\newcommand{\floor}[1]{\ensuremath{\left \lfloor #1 \right \rfloor}}
\newcommand{\dx}{\ensuremath{\text{d}}}
\newcommand{\ind}{\ensuremath{\mathbf{1}}}
\newcommand{\basisvec}{\ensuremath{\mathbf{e}}}
\newcommand{\vect}[1]{\ensuremath{\mathbf{#1}}}
\newcommand{\defn}{\ensuremath{\coloneqq}}
%\newcommand{\zeroone}{\ensuremath{\operatorname{0-1}}}
\renewcommand{\epsilon}{\varepsilon}


\DeclareMathOperator{\sign}{sign}
\DeclareMathOperator{\argmin}{argmin}

\title{Statistical Rethinking: Notes and Selected Exercises}

\author{Somnath Sikdar}
\date{\today}

\begin{document}
\maketitle
\tableofcontents

\setcounter{chapter}{1}

\chapter{Small Worlds and Large Worlds}

\begin{exer}
Suppose there are two species of panda bear. Both are equally common in the 
wild and live in the same places. They look exactly alike and eat the same 
food, and there is yet no genetic assay capable of telling them apart. They 
differ however in their family sizes. Species $A$ gives birth to twins 10\% of 
the time, otherwise birthing a single infant. Species $B$ births twins 20\% of 
the time, otherwise birthing singleton infants. Assume these numbers are 
known with certainty, from many years of field research.
Now suppose you are managing a captive panda breeding program. You have a new 
female panda of unknown species, and she has just given birth to twins. What 
is the probability that her next birth will also be twins?
\end{exer}
\begin{solution}
We have to estimate $\Prone{\text{twins again} \vert \text{twins before}}$. 
We may write this conditional probability as follows:
\begin{align} \label{eqn:twins}
    \Prone{\text{twins again} \vert \text{twins before}} & = 
        \Prone{\text{twins again} \vert A, \text{twins before}} \cdot 
        \Prone{A \vert \text{twins before}} + \nonumber \\
        & \quad \quad 
        \Prone{\text{twins again} \vert B, \text{twins before}} \cdot 
        \Prone{B \vert \text{twins before}} \nonumber \\
        & = \Prone{\text{twins again} \vert A} \cdot 
        \Prone{A \vert \text{twins before}} + \nonumber\\
        & \quad \quad 
        \Prone{\text{twins again} \vert B} \cdot 
        \Prone{B \vert \text{twins before}}
\end{align}

The last equality follows, since when we condition on any given species, 
the probability of having twins again is independent of whether twins 
were born before. Strictly speaking, this is also an assumption but a reasonable 
one. As such, we do not use the ``before'' and ``again'' 
qualifiers. To evaluate the last expression, we calculate 
$\Prone{ \cdot \vert \text{twins}}$ for both species $A$ and $B$.  Using 
Bayes' Theorem, we obtain:
\begin{align*}
    \Prone{A \vert \text{twins}} & = 
        \frac{\Prone{A \vert \text{twins}} \cdot \Prone{A}}
             {\Prone{A \vert \text{twins}} \cdot \Prone{A} + \Prone{B \vert \text{twins}} \cdot \Prone{B}} \\
        & = \frac{0.10 \times 0.5}{0.5 \times (0.1 + 0.20)} \\
        & = \frac{1}{3}.
\end{align*}

A similar calculation yields $\Prone{B \vert \text{twins}} = \frac{2}{3}$. The 
final expression in~(\ref{eqn:twins}) evaluates to 
\[
0.10 \times \frac{1}{3} + 0.20 \times \frac{2}{3} = \frac{1}{6} = 0.17.
\]
This probability lies between $0.10$ and $0.20$ as expected and lies closer 
to $0.20$ than $0.10$. This is also clear since if the panda has had twins before, 
it is more likely to be of species $B$ than species $A$. Hence it is more likely
that she will birth twins again.
\end{solution}


\end{document}

